\documentclass[12pt,a4paper,openany,oneside]{book}

\usepackage{hyperref}
\usepackage[italian]{babel}


%\usepackage[latin1]{inputenc} %Windows
\usepackage[utf8x]{inputenc} %Linux
%\usepackage[applemac]{inputenc} %Mac

\usepackage{graphicx}
\usepackage[font=small,labelfont=bf,tableposition=top]{caption}

\usepackage{listings}
  \usepackage{courier}
 \lstset{
         language=C++,
         basicstyle=\footnotesize\ttfamily,
         numbers=left,           
         numberstyle=\tiny,       
         numbersep=5pt,        
         tabsize=5,                 
         extendedchars=true,         
         breaklines=true,
         keywordstyle=\textbf,           
         stringstyle=\color{white}\ttfamily,
         showspaces=false,       
         showtabs=false,           
         xleftmargin=17pt,
         framexleftmargin=17pt,
         framexrightmargin=5pt,
         framexbottommargin=4pt,
         showstringspaces=false          
 }
 
 \addto\captionsitalian{%
 	\renewcommand{\lstlistingname}{Codice}}

\setcounter{tocdepth}{3} % fa apparire le subsubsection nell'indice
\setcounter{secnumdepth}{3} % e le numera


\usepackage{amsmath}


\usepackage{framed}

\begin{document}

%Inserisce qui il frontespizio
\begin{titlepage}
\centering 
\includegraphics[width=2.434cm,height=2.565cm]{Images/university_logo.png}

\bigskip

{\Large \textbf{UNIVERSIT\`A DEGLI STUDI DI CATANIA}}

{\scshape
\large
Dipartimento di Matematica e Informatica
}

{\scshape
\normalsize
Corso di Laurea Triennale in Informatica
}

\bigskip


\hrule


\bigskip


\bigskip


\bigskip


\bigskip

{\itshape
\large
Nome Autore
\par}


\bigskip


\bigskip


\bigskip


\bigskip

{\centering
\Large
Titolo della Tesi
\par}


\bigskip


\bigskip


\bigskip


\bigskip


\bigskip


\bigskip


\begin{minipage}[b]{8 cm}
\hrule

\bigskip

{\centering\scshape 
Relazione Progetto Finale
\par}


\bigskip

\hrule
\end{minipage}
\bigskip


\bigskip


\bigskip


\bigskip


\bigskip


\bigskip


\bigskip


\bigskip


\bigskip


\bigskip


\bigskip

{\raggedleft
Relatore: ---- \\
Correlatore: ----
\par}


\bigskip


\bigskip


\bigskip


\bigskip

\hrule

\bigskip

{\centering
Anno Accademico 20xx - 20xx
\par}
\end{titlepage}
 %<- richiama il file "parts/titlepage.tex" che contiene il frontespizio

\bibliographystyle{unsrt}
\title{Titolo Della Tesi}
\author{Nome Autore}

%Inserisce qui l'abstract
\chapter*{Abstract} %l'asterisco dopo chapter serve per visualizzare il capitolo come "non numerato"
L'abstract va inserito qui.


%Genera l'indice
\tableofcontents

\chapter{Introduzione}
Qui va inserito il capitolo 1. Un esempio di figura è mostrato in \figurename{~\ref{fig:immagine}}. Questo è un esempio di citazione \cite{iplab}. Un esempio di listato è mostrato in Codice~\ref{lst:multicorrelation}. Un esempio di tabella è mostrato in \tablename~\ref{tab:gen}.

\begin{figure}[t] %[t] per inserire la figura in cima alla pagina
	\includegraphics[width=1\linewidth]{Images/iplab} %width=1\linewidth per scalare l'immagine alle dimensioni opportune. E' possibile ridurre le dimensioni dell'immagine inserendo un numero minore di 1
	\caption{Questo è un esempio di didascalia.}
	\label{fig:immagine} %per poter richiamare l'immagine nel testo
\end{figure}


\begin{lstlisting}[caption={Didascalia},label=lst:multicorrelation]
double multicorrelation_ncc(template_obj,candidate_obj) {
double similarity = ncc(template_obj,candidate_obj);
if (similarity<t) {
split template_obj into blocks: template_obj[9];
split candidate_obj into blocks:  candidate_obj[9];
similarity=0;
for (i=0; i<9; i++)
similarity+=ncc(template_obj[i],candidate_obj[i]);
similarity/=9;
}
else
return similarity;
}
\end{lstlisting}

\begin{table}[ht]
	\begin{center}
		\caption{Tabella Generica}\label{tab:gen}
		\begin{tabular}{lcccccc}
			
			& & & & & & \\ \hline
			\multicolumn{1}{|c}{}  & \multicolumn{5}{|c|}{\bf Y} &
			\multicolumn{1}{c|}{} \\
			
			\multicolumn{1}{|c}{\bf X} & \multicolumn{1}{|c}{$1$} & $\ldots$ & $j$ & $\ldots$ &
			\multicolumn{1}{c|}{$k$} & \multicolumn{1}{c|}{$\Sigma_{j=1}^k{X_{ij}}$} \\ \hline
			
			\multicolumn{1}{|c}{$1$} & \multicolumn{1}{|c}{$X_{11}$} & $\ldots$ & $X_{1j}$ &
			$\ldots$ & \multicolumn{1}{c|}{$X_{1k}$} &
			\multicolumn{1}{c|}{$m$}\\
			
			\multicolumn{1}{|c}{$\vdots$} & \multicolumn{1}{|c}{$\vdots$} &  & $\vdots$ &  &
			\multicolumn{1}{c|}{$\vdots$} &
			\multicolumn{1}{c|}{$\vdots$}\\
			
			\multicolumn{1}{|c}{$i$} & \multicolumn{1}{|c}{$X_{i1}$} & $\dots$ & $X_{ij}$ &
			$\ldots$ & \multicolumn{1}{c|}{$X_{ik}$} &
			\multicolumn{1}{c|}{$m$} \\
			
			\multicolumn{1}{|c}{$\vdots$} & \multicolumn{1}{|c}{$\vdots$} &
			& $\vdots$ &  & \multicolumn{1}{c|}{$\vdots$} & \multicolumn{1}{c|}{\vdots} \\
			
			\multicolumn{1}{|c}{$n$} & \multicolumn{1}{|c}{$X_{n1}$} & $\dots$ & $X_{nj}$ &
			$\ldots$ & \multicolumn{1}{c|}{$X_{nk}$} & \multicolumn{1}{c|}{$m$}\\ \hline
			
		\end{tabular}
	\end{center}
\end{table} %<- richiama il file "parts/capitolo1.tex" che contiene il primo capitolo
%Si consiglia di creare un nuovo file per ogni capitolo

\chapter*{Conclusione} %l'asterisco dopo chapter serve per visualizzare il capitolo come "non numerato"
\addcontentsline{toc}{chapter}{Conclusione} %per fare inserire il capitolo nella tabella dei contenuti
 %<- richiama il file "parts/conclusione.tex" che contiene la conclusione

%Inserisce la bibliografia
\newpage
\addcontentsline{toc}{chapter}{Bibliografia}
\bibliography{bibliography}
\end{document}